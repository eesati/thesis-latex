\chapter{Introduction and Motivation}
\label{thesis:introduction}



Internet of Things (IoT) is transforming the world of things into an autonomous world. It has a high impact in many industries such as manufacturing, transportation, automotive, consumer goods, and even healthcare will never be the same
once IoT is applied. Thanks to the advances of the underlying technologies, IoT devices are being equipped with more powerful processors and are able to effectively distribute the processing load to other cores. This offers the opportunity of to run more complex tasks in the IoT devices. In our study we are going to use ESP EYE (reference) which is equipped wit camera and microphone which comes with plenty of storage with an 8 Mbyte PSRAM and a 4 Mbyte flash. 
However IoT comes with a number of challenges or gaps  that still need to be improved such as: security issues, centralization and the vulnerability to attacks. 
On the other hand blockchains as a decentralized network is able to hold to records and transactions in blocks secured by cryptography. We can see that the advantages of blockchain converge with the disadvantages of IoT which can eventually transform how information is processes and stored. On the other hand Artificial Intelligence (AI) has a significant role as an accurate analysis of data in real time. However with the design and development of an efficient data analysis tool  using AI comes with challenges : centralized architecture, security issues and the decisions made from AI are not transparent and recorded. Therefore integrating the blockchain with AI has the ability to produce a robust technology to resolve the several AI issues. AI as a black box box lacks transparency, therefore with the transparency of blockchain with sharing the data in many nodes in a subsequent order provides a clear way for tracking back the data to the AI decision process.   

Therefore the convergence of the three domains will bring a successful synergy that will transform how data is being processed, analyzed and stored. In this study we aim to explore the three mentioned emerging paradigms that will influence the future. Therefore the main goal of this research is the design and development of a Blockchain-enabled Intelligent IoT Architecture with Artificial Intelligence that will provide an efficient implementation of a use case while incorporating the three domains with the most state of the art technologies and applications. 
There are a number of attempts that shed light on the benefits of converging IoT, blockchain and AI (reference) but most of them are either reviews or explorations but far away from a concrete implementation of a use case (ref2). A more close research in the this matter is an attempt (ref) which proposes a BlockIoTIntelligence architecture as they call it that converges the blockchain, AI for IoT whose aim is to achieve the goal of big data analysis, security and centralization issues of IoT applications such smart city, healthcare and smart transportation. With their findings they claim that BlockIoTIntelligence can mitigate the existing challenges and obtain high accuracy with a reasonable latency and security in the decentralized way. 
If narrowing down the scope to blockchain and IoT we see a slightly more research who attempt to close the gaps of IoT by removing the centralized control with the help of blockchain (refmyown).
For example [ref3]attempted to find out the security and privacy in IoT  gaps that could be filled with the help of the blockchain to ensure the reliability and availability of the data. 

Having had a closer look on the research that shed light on the synthesis of the three domains we see a number of proposals and architecture but yet we do not see a real use case where the three technologies can complement each other. 
Therefore the main goal of this project is to design and implement an IoT surveillance system with blockchain and AI to support access control with facial detection and recognition. Therefore the use case we are designing and implementing provides an efficient way of converging blockchain, AI and IoT with the most current state-of-the-art approaches. 
We are entering an age where surveillance is becoming a norm and facial recognition technologies are hunting the streets. The roll out of facial recognition technologies is expected to become more prominent and an important step in improved security. On the other side, major data breaching has become common place and people worry about the type of personal information held by organizations. In the midst of the COVID-19 pandemic, attention have turned to facial recognition technology as a way to combat the spread of the virus.

Before introducing our own architecture an interesting use case that resembles  our architecture is an implementation of a camera based sensor for monitoring room occupancy. Their architecture employees AI and IoT. This use case was designed and implemented by a group of students at FHNWS who employed a raspberry pi equipped with camera and lora gateway. The role of camera is to just take pictures and after that some machine learning algorithm will analyze the image and find the number of people in the room. Once done the data is then send to a LoraWan Server. To achieve that they have attached a LoraWan antenna to the RPI and eventually with a front end they can monitor the room. So face detection happens in the RPI, the algorithm counts the number of people in the image and sends it to the Lorawan Backend Server. From here we can conclude that the issue of centralisation is still open, data is being stored in a database, security issues are not tackled enough. Besides there is also the need for a RPI to be placed in a room as the camera is attached to it and that needs power to run.

Hence to combat the many issues we have discussed and found in the above use case we will be using a low powered IoT device named ESP EYE which with the help of AI running on it will detect and recognize the individuals and grant or reject access. In the due time the ESP EYE will capture images of individuals and forwards them to the blockchain where it will be stored. The proposed architecture and implementation paves the way towards a strong architecture that fills the many gaps of the three domains. Therefore the main goal of this project is to design and implement an IoT-based surveillance system with blockchain and AI to support access control with facial detection and recognition. 
The Face detection and recognition is done in the ESP EYE itself with the help of two machine learning algorithms. For the face detection we employee the MTCNN (Multi-task Cascaded Convolutional Neural Networks ). With the help of FRMN ( Human Face Recognition Model) the person will be recognized. 



% ~\cite{morin}.
%~\cite{what_is_lora}.

 transceiver\footnote{https://www.semtech.com/products/wireless-rf/lora-transceivers/sx1276} found on regular LoRa devices and gateways. 


 Docker\footnote{https://www.docker.com/}) and executed in the cloud environment. 

\section{Description of Work}
This work gives a general introduction to LoRa, LoRaWAN, and its applications, provides a detailed description of the LoRa physical layer, and gives an overview of existing software implementations of the LoRa PHY. 
Furthermore, there are three main research contributions. 
First, this thesis provides an architectural overview, specification of the system, and the implementation details of a C-RAN for LoRa.
Second, the network related requirements are evaluated experimentally. 
We develop a simple application protocol using an underlying LoRa network, which is not yet fully compatible with the LoRaWAN. 
In the application protocol evaluation, a physical IoT device sends data packets towards the LoRa gateway.
Some packets require an acknowledgment, therefore, a node has to wait a few seconds for an acknowledgement established by the gateway.
Typically, when the sender requires an acknowledgement for a given packet, but no acknowledgment is received, the same packet will be repeated until the acknowledgement finally arrives.
We investigate network utilization and effects of network and processing delays on the LoRa C-RAN performance.
Third, as the LoRa PHY is closed source, there is no official documentation on how the LoRa PHY is implemented.
The existing implementations are all reverse engineering attempts with a various degree of success. 
They all focus on decoding LoRa signals transmitted by a regular hardware. 
In the successful LoRa C-RAN, it is required to successfully decode signals and also encode downstream LoRa signals. 
To achieve this, we extend an existing experimental uplink signal encoder with the ability to generate downlink signals in software.

\section{Thesis Outline}
The rest of the thesis is structured in the following way.
Chapter~\ref{chap:lora_and_lorawan} provides an introduction to LoRa and LoRaWAN.
LoRa is also compared against other wireless technologies and an in-depth explanation of LoRa signals is given. 
The modulation scheme and key factors such as spreading factor, coding rate, and packet structure are introduced.

In the following chapter, an overview of current software defined radio implementations for LoRa is given.
The chapter discussed various implementations and their level of sophistication. 
The GNU Radio framework is introduced as well.

In chapter~\ref{chap:cran_in_cellular}, the C-RAN architecture for cellular networks is introduced to show the steps needed to move from a traditional setup towards a C-RAN architecture.
Several benefits of C-RAN are discussed as well.

The next chapter focuses on the C-RAN for LoRa. 
The architectural overview, system specification as well as the implementation details for all involved components are provided. 
The C-RAN experiment is specified, in which network utilization, network delay, and processing delay are investigated.
Finally, the chapter presents the experimental evaluation of the system.

Chapter~\ref{chap:lora_tools} presents various tools that were developed during this thesis for encoding and signal visualization that may be helpful in future developments of LoRa signal processing in software.

The second-last chapter discussed the future work in C-RAN for LoRa. It also lists some limitations of the current C-RAN architecture that can be improved and developed in the future.

Finally, the last chapter summarizes and concludes this work.





